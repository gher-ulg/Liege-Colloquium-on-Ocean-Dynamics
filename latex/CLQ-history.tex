\documentclass[12pt,a4paper,oneside,final]{article}
\usepackage[utf8]{inputenc}
\usepackage[english]{babel}
\usepackage{amsmath}
\usepackage{amsfonts}
\usepackage{amssymb}
\usepackage{graphicx}
\usepackage{lmodern}
\usepackage{url}
\usepackage{hyperref}
\author{C. Troupin}
\title{50 years of Colloquium on Ocean Dynamics}
\begin{document}

\section{Data}

Participants'information was obtained from the Special Edition available either in electronic or in paper format and organized in 5-column, tab separated ascii files. Such a format allows for an easy ingestion by almost any spreadsheet tool. Every file contains: 
\begin{enumerate}
\item The participant name,
\item their surname initials (necessary for the distinction of authors with the same name),
\item the institute,
\item the city,
\item the country.
\end{enumerate}
While we tried to stay as close as possible to the origin lists, the country names were adapted in order to fit with the existing countries in 2017, using the information of the city when necessary. For example, \textit{Republic of Serbia} instead of \textit{Yugoslavia}; \textit{Estonia} instead of \textit{U.S.R.R}.
The institute were not entered in a consistent way from year to year. However, almost no editing was performed since the main goal was to determinate the approximate coordinates using the combination of institution, city and country. 

\subsection{Availability}

All the data files are stored in a version control system (VCS) and made available from \url{https://www.github.com}. Readers are encouraged to submit their bug report either through github pull request mechanism or by direct email to the author. 

\section{Processing}

This section describes the tools developed to produce the visualization and results presented in the next Section. 
All the data processing was performed using Python-3.6. As for the data, the tools are made available from the same Github repository.

\subsection{Geolocalisation}

The objective was to obtain the coordinates of each participant's institution to prepare density maps. In most of the cases the coordinates are properly found using the institute name, city and country, but a difficulty arises when an institute name has changed in the past. To enhance the probability of success, different search engine are used until the coordinates are returned. No test was performed to analyze if the different engines provided different results as the goal is to have approximate coordinates. For each institute, the coordinates obtained by geolocalisation were stored in order to avoid unnecessary calls to the search engine once the place was already located.

\subsection{Country counting}

The goal is to obtain a dictionary where the keys are the country names and the values are the number of participants during a selected period.


\section{Results and visualizations}

\subsection{Choropleth map}

3rd colloquium:

The increasing success of the First and Second Colloquia on Ocean Hydrodynamics organized at the University of Liège commanded that they be continued. In this course; the organizers were encouraged by the sustained interest, advice and support of Professor M. Dubuisson, Recteur de l'Université de Liège et Président du Centre Belge d'Océanographie, all all the members of the Scientific Committee for Ocean Research at the University oif Liège and its Oceanographic Center in Corsica who deserve more gratitude than words, in this brief introduction, will ever convey.

It is a great joy for the Organizing Committee to realize that, in the same time, the continuation of the Colloquia was deeply desired by all the participants and that their gratifying determination was indeed answerable for making, from now on, the Liège Colloquium on Ocean Hydrodynamics an annual meeting.

There is not better prospect of success than to find in the list of participants, representing several European countries and the United States, the names of many of the most famous hydrodynamicists and physical oceanographers; some of whom attended the Colloquium last year and the year before and do not conceal -- so much they feel at home and among friends in Liège --, that they considered themselves as permanently registered. 



\end{document}